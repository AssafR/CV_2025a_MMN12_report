% --------------------------------------------------------------
% This is all preamble stuff that you don't have to worry about.
% Head down to where it says "Start here"
% --------------------------------------------------------------
 
\documentclass[11pt,a4paper]{article}

\usepackage{pgfplots} % For plotting
\usepackage{pgfplotstable}
\usepackage{enumitem} % Custom list handling
\usepackage{booktabs} % For better table styling
\usepackage{listings}
\usepackage{xcolor}

\pgfplotsset{compat=1.18} % Ensure compatibility
\usepackage{caption} % For \captionof
\usepackage{mathrsfs,amsmath}   %The amsmath package is included for \xrightarrow
\usepackage[top=2cm,bottom=2cm,left=2.5cm,right=2cm]{geometry}
\usepackage{array} % For better column formatting
\usepackage{ragged2e} % For \RaggedRight to ensure proper alignment
% \usepackage{polyglossia}
\usepackage{framed}


\usepackage[english, bidi=basic, layout=sectioning.tabular]{babel}
\babelprovide[import,main]{hebrew}
\babelfont{sf}
          [Ligatures={Common,Discretionary,TeX}]{Libertinus Sans}
\babelfont{tt}
          [Ligatures=TeX]{Libertinus Mono}


\setmainfont{David CLM}[Script=Hebrew]
%\setmainfont{Noto Sans Hebrew}[Language=Hebrew, Script=Hebrew]

\babelfont[hebrew]{rm}[Script=Hebrew]{David CLM}
\usepackage{fontspec}

\newfontfamily\hebrewfont[Script=Hebrew]{David CLM} %This font will work in Overleaf, Linux and Mac installs, if you're running windows you'd need to pick the right ones
\newfontfamily\hebrewfontsf[Script=Hebrew]{Miriam CLM}
\newfontfamily\englishfont{Times New Roman} % English font
%\newfontfamily\englishfont{TeX Gyre Termes} % Robust alternative to Times New Roman
          
% [Ligatures={Common,Discretionary,TeX}]{Libertinus Serif} % Or any font that supports Hebrew.

\newenvironment{solution}{\begin{proof}[Solution]}{\end{proof}}

\newcommand{\EN}{\selectlanguage{english}}
\newcommand{\HE}{\selectlanguage{hebrew} \vspace{0pt}}
\newcommand{\ToEn}[1]{\EN{#1}\HE}

% Define the custom itemize environment for hebrewlist
\newlist{hebrewlist}{itemize}{1} % Create a new list type
\setlist[hebrewlist,1]{%
    label=\fixNumbering{\thehebrewCounter}, % Use Hebrew counter with tightly bound dot
    ref=\thehebrewCounter, % Reference format (if needed for cross-referencing)
    leftmargin=2em, % Adjust the left margin
    labelsep=1em, % Adjust spacing between the label and the text
    itemsep=0.5\baselineskip, % Adjust spacing between items
    before=\usecounter{hebrewCounter} % Enable the Hebrew counter
}

% Define the macro to switch languages
\newcommand{\switchlang}[2]{%
    \foreignlanguage{#1}{#2} % Switch to the specified language for the given text
}


% Example: Switch to Hebrew for this section
%    \switchlang{hebrew}{זוהי טקסט בעברית}


%--------------------------------


\newcommand{\CreateCustomTable}[2]{%
    \begin{table}[ht!]
    \centering
    \caption{#1}
    \EN
    \pgfplotstabletypeset[
        col sep=comma, % Specifies that the CSV uses commas as the column separator
        header=true, % Use the first row as the table header
        columns/extractor/.style={string type}, % Format 'extractor' column as text
        columns/classifier/.style={string type}, % Display 'classifier' with 4 decimal points
        columns/accuracy/.style={fixed, precision=4}, % Display 'Repeatability' with 4 decimal points
        columns/AUC/.style={fixed, precision=4}, % Display 'LocalizationError' with 4 decimal points
        every head row/.style={before row=\hline, after row=\hline}, % Horizontal lines above and below header
        every last row/.style={after row=\hline} % Add a horizontal line after the last row
    ]{#2}
    \end{table}
}


% Add a figure
\newcommand{\CreateFigure}[2]{%
    \begin{figure}[ht!]
        \centering
        % Adjust the spacing before and after the caption
        \setlength{\abovecaptionskip}{2pt} % Reduce space above the caption
        \setlength{\belowcaptionskip}{2pt} % Reduce space below the caption
        \includegraphics[width=0.56\textwidth]{#1} % Increase the width to 55% of text width
        \caption{#2} % Use the second parameter as the caption
        \label{fig:#1} % Optional: Use the filename as the label (or adjust as needed)
    \end{figure}
}



\lstset{
    basicstyle=\ttfamily\footnotesize,  % Use a terminal font and reduce the size
    frame=single,                       % Add a frame around the code
    xleftmargin=1em,                    % Adjust left margin
    showspaces=false,                   % Do not show spaces explicitly
    showstringspaces=false,             % Do not show spaces in strings
    breaklines=true,                    % Enable line breaking
    backgroundcolor=\color{gray!10},    % Light gray background
}



% --------------


% Redefine the itemize bullets globally
\renewcommand\labelitemi{$\bullet$}
\renewcommand\labelitemii{$\circ$} % Open circle for the second level
\renewcommand\labelitemiii{$\star$} % Star for the third level


% Set up a Hebrew-compatible font
\setmainfont{David CLM} % Replace with another Hebrew-compatible font if needed

% Define Hebrew labels manually using a custom counter
\newcounter{hebrewCounter} % Create a new counter
\renewcommand{\thehebrewCounter}{% Map the counter value to Hebrew letters
    \ifcase\value{hebrewCounter}
    \or א
    \or ב
    \or ג
    \or ד
    \or ה
    \or ו
    \or ז
    \or ח
    \or ט
    \or י
    \else ?
    \fi
}

% Force Hebrew letter and period to act as a single unit
\newcommand{\fixNumbering}[1]{\makebox[0pt][r]{#1\kern-0.3em.}} % Use \kern to remove space

\begin{document}

 
% --------------------------------------------------------------
%                         Start here
% --------------------------------------------------------------
\HE 
\title{ממ"ן 12 - ראייה ממוחשבת}
\author{אסף רזון}
\maketitle
בתרגיל זה  3 חלקים. 

\section*{שאלה {1}}
כצעד ראשון, התבקשנו לחלק את הנתונים לנתוני אימון 
\ToEn{train-set} כאשר
\ToEn{60\%} מהתמונות ישמשו לאימון, \ToEn{20\%} לוולידציה ו- \ToEn{20\%} לבדיקה.
\medskip
\subsection*{חלוקה של הדאטה למחיצות}
החלוקה נעשתה תוך שימוש בפונקציית \ToEn{train\_test\_split} מהספרייה, הקוד נמצא בקובץ \ToEn{split\_train.py}.\newline
לאחר יצירת רשימת כל הקבצים, נעשית הגרלת חלוקה לשלוש המחיצות
\ToEn{Train, Validation, Test}
לפי ההסתברויות שהתבקשנו. לאחר מכן הקבצים מופנים ל-3 ספריות בשמות המתאימים.

\subsection*{שיטת העבודה}
ניתן לחלק את ה-\ToEn{pipeline} לכמה שלבים, שכל אחד מהם קורא פלט מדיסק, מעבד אותו, ושומר לדיסק את השלב הבא.
\newline
כך ניתן לפתח כל שלב בנפרד ולחזור אחורה בלי להתחיל בכל פעם מההתחלה.

השלבים הם:
\begin{enumerate}[start=0]
\item חלוקה של הדאטה (תמונות) למחיצות
\item הפעלה של \ToEn{Feature Extraction} על כל התמונות ושמירת הפיצ'רים בלבד
\item פעולת קוונטיזציה (\ToEn{Vector Quantization})  - אימון \ToEn{K-Means} על מרחב הפיצ'רים שהתקבל בשלב הקודם ושמירת המודל. מספר המרכזים שבחרתי בשלב זה הוא 128.
\item שלב תרגום ע"י \ToEn{Bag Of Features} - לכל תמונה המאופיינת כעת ע"י סט פיצ'רים, חישוב האשכול \ToEn{(Cluster)} שאליו שייך כל פיצ'ר. כעת הייצוג של התמונה הופך להיות היסטוגרמה - לכל אשכול k מהו מספר המאפיינים ששייך לו.\newline
הייצוג הזה של תמונה כהיסטוגרמה נשמר לדיסק.
\item אימון מסווג אשר מקבל כקלט בשלב האימון זוגות של היסטוגרמה+מחלקה , ולומד לסווג היסטוגרמה למחלקה
\item הפעלת המסווג על מחיצות \ToEn{test, validation} וחישוב המטריקות המתאימות להערכת הביצועים
\end{enumerate}
\par
נשים לב כי ה-\ToEn{pipeline} הזה זהה לגמרי בין שאלות 1,2 - ההבדל העיקרי הוא איזה אלגוריתם מבצע את פעולת \ToEn{Feature Extraction}.\par
בשאלה מס' 1 השתמשתי באלגוריתם \ToEn{Orb}, אשר מחזיר פיצ'רים בגודל 32 - במספר שאינו ידוע מראש אך חסום.\newline
בשאלה 2 השתמשתי בשכבות הראשונות של רשת נוירונים מסוג \ToEn{VGG16} ללא השכבות האחרונות שלה - המוצא של שכבות אלה הוא 64 פיצ'רים בגודל 512, שכל אחד מהם מתאים לסביבה של ריבוע אחד בתמונה המקורית ומתאר אותו ואת סביבתו (ברשת קלאסית בגודל \ToEn{64x64} כל סביבה כזאת היא פיקסל).

\subsection*{נקודות חשובות בפתרון של שני הסעיפים האלה}
\begin{itemize}
    \item מספר הפיצ'רים שמחזירה רשת הנוירונים הוא קבוע. אבל מספר הפיצ'רים שמוחזרים ע"י אלגוריתם כגון \ToEn{Orb} יכול להיות קטן יותר או גדול יותר מהמספר שהחלטנו לקבוע (כאן בחרנו לקבוע קבוצה בגודל 500).\newline
    במקרה זה יש לחתוך או להשלים את הפיצ'רים למספר קבוע, כיוון שהיישום הבסיסי של היסטוגרמה כזו מניח מספר קבוע (וסכום ערכי ההיסטוגרמה יהיה שווה למספר הזה). \newline
    לכן חייבים לבצע התאמה למספר קבוע של פיצ'רים.
    \begin{itemize}
    \item  
אם המספר קטן מדי, משלימים את הקבוצה באמצעות אפסים (כלומר פיצ'רים שערך כולם הם אפס).
\item אם המספר גדול מדי, ממיינים את הפיצ'רים לפי ערך ה- \ToEn{Response} ושומרים רק את בעלי הערכים הגבוהים עד לגודל הקבוצה הרצוי.
\item ערך זה מייצג "איכות" או "בטחון" של האלגוריתם בנקודת המפתח שהתגלתה. הנקודות שבהן ערך זה גבוה הן הבולטות יותר. לכן הגיוני להשתמש בנקודות שיש להן ערך תגובה גבוה (עד למספר הנדרש) ולהתעלם מן האחרות.
\item אם אכן יהיו הרבה "השלמות" של פיצ'רים ריקים (ערך 0) בסט האימון, סביר להניח שלפחות אחד מהאשכולות שיחושב באלגוריתם \ToEn{k-means} יהיה קרוב מאד אל הנקודה של ראשית הצירים, כשאליה ימופו כל הקואורדינטות הריקות הנ"ל.
    \end{itemize}
\item המידע הטבלאי נשמר באמצעות \ToEn{Pandas DataFrame} בפורמט \ToEn{feather} המאופיין ביעילות קריאה גבוהה. 
\item המידע הבינארי, כגון סריאליזציה של מודלים, נשמר בפורמט \ToEn{pickle}.
\item לצורך בניית המסווג בשלב הסופי, ניסיתי מספר אפשרויות.
    \begin{itemize}
        \item שתי וריאציות של מסווג \ToEn{AdaBoostClassifier}
        \item שתי וריאציות של מסווג \ToEn{XGBClassifier}
    \end{itemize}
\end{itemize}

\subsection*{הרצת השאלה}
לצורך הרצת פתרון השאלה יש להריץ את הפקודה:
\EN
\begin{framed}
\begin{verbatim}
cd src
python mmn12_q1.py
\end{verbatim}
\end{framed}
\HE
במצב ברירת המחדל ללא פרמטרים, ההרצה תתחיל משלב 1 כפי שמפורט ב"שיטת העבודה".
לחילופין ניתן להריץ החל משלב מתקדם יותר, למשל:
\EN
\begin{framed}
\begin{verbatim}
python mmn12_q1.py --stage=3
\end{verbatim}
\end{framed}
\HE

\section*{שאלות שנשאלנו}
מה הפרמטרים האופטימליים של המסווג?\newline
יש להציג את הביצועים ע"י עקומות 
\ToEn{ROC}, חישוב ה-\ToEn{AUC} וגם \ToEn{confusion matrix} ו- \ToEn{precision-recall} 
\section*{תשובות לשאלות}
הטבלה:
\HE
על-פי מדד ה-\ToEn{AUC}, נראה שהמסווג המוצלח ביותר עבור \ToEn{Orb} הוא \ToEn{ADABOOST2}. הפרמטרים שבהם השתמשתי עבורו הם:
\EN
\begin{lstlisting}
'ADABOOST2':
AdaBoostClassifier(
    estimator=DecisionTreeClassifier(max_depth=2), 
    n_estimators=20,  # Number of boosting rounds
    learning_rate=0.2,  # Learning rate
    random_state=42  # Seed for reproducibility
)
\end{lstlisting}
\CreateCustomTable{ביצועי המסווגים השונים}{data/results_extractor_Orb.csv}
%\clearpage

\HE
\newpage
והגרפים המתאימים עבור מחיצת ה-\ToEn{Test} הם:

\CreateFigure{images/roc_curve_global_Orb_ADABOOST2_test.png}{עקום \ToEn{ROC}}
\CreateFigure{images/prc_curve_global_Orb_ADABOOST2_test.png}{עקום \ToEn{Precision-Recall}}
\CreateFigure{images/confusion_matrix_Orb_ADABOOST2_test.png}{מטריצת הבלבול}
\clearpage
\newpage
\HE
\section*{שאלה {2}}
בשאלה 2 השתמשתי באותו ה- \ToEn{pipeline} , עם הבדלים מעטים בלבד: שימוש בטנזורים (התוצר שיוצא מתוך הרשת), מימדים שונים במעט, וכו'.

גם כאן בשלב האחרון בדקתי את אותם מסווגים ואלה התוצאות שלהם:
\CreateCustomTable{ביצועי המסווגים השונים}{data/results_extractor_VGG.csv}

כל ארבעת המסווגים הגיעו לתוצאות דומות, אם כי כאן המוצלח ביותר היה \ToEn{XGB2} , אשר הפרמטרים שלו הם:
\EN
\begin{lstlisting}
'XGB2': 
XGBClassifier(
    objective='multi:softmax',  # Multiclass classification
    num_class=8,  # Number of classes
    max_depth=3,  # Tree depth
    learning_rate=0.03,  # Learning rate (eta)
    n_estimators=10,  # Number of trees
    random_state=42  # Seed for reproducibility
),
\end{lstlisting}
\HE
\newpage
ואילו הגרפים שמתאימים לתוצאות שלו הם:

\CreateFigure{images/roc_curve_global_VGG_XGB2_test.png}{עקום \ToEn{ROC}}
\CreateFigure{images/prc_curve_global_VGG_XGB2_test.png}{עקום \ToEn{Precision-Recall}}
\CreateFigure{images/confusion_matrix_VGG_XGB2_test.png}{מטריצת הבלבול}
\clearpage

\newpage
\HE
\section*{שאלה {3}}
בשאלה 3 התבקשנו לממש רשת נוירונים מסוג קונבולוציה ולאמן אותה כדי לסווג את הדאטה ל-8 המחלקות שלו.\newline
להרצה:
\EN
\begin{framed}
\begin{verbatim}
cd src
python mmn12_q3.py
\end{verbatim}
\end{framed}
\HE

כצעד ראשון התבקשנו להגדיר \ToEn{dataloader} ולהציג מספר דוגמאות מכל מחלקה.
תמונת הדוגמאות נוצרת בשלב מס' 1 של השאלה השלישית. 
\CreateFigure{images/class_sample_images.jpg}{דוגמאות מתוך סט האימון עבור כל אחת מ-8 המחלקות שאותן נדרשים לסווג}
\newline

\subsection*{שיטת העבודה}
על-מנת להגיע לתוצאות הנדרשות (90 אחוזי דיוק) נדרשו החלטות וניסויים רבים.

החלטתי לבחור בטכניקה של \ToEn{Fine-Tuning} על רשת קיימת.
במצב כזה, הרשת שאומנה על כמויות גדולות של דאטה (תמונות) למדה לחלץ פיצ'רים רלוונטיים למשימה של זיהוי וסיווג תמונות. נשאר רק לבצע כוונון עדין יותר עבור המטרה הספציפית הזאת.\newline
כדי לבצע זאת "מקפיאים" את השכבות הראשונות ברשת, כלומר מונעים מהן להתעדכן גם בשלב האימון (לא מתבצע עליהן \ToEn{Backpropagation} וחישובי גראדיינטים). לעומת זאת השכבות האחרונות, בד"כ שכבת \ToEn{Fully Connected}, כן לומדות את הדאטה החדש ומתכווננות לפיו. \newline
יתרונות נוספים של גישה זאת:
\begin{itemize}
    \item הרשת גדולה ולכן יכולה להתמודד עם משימות מתוחכמות
    \item נשים לב שהדאטה שלנו קטן יחסית, ולכן לא מספיק לאימון מלא של רשת גדולה. באופן כללי מספר פרמטרים גדול דורש כמות גדולה יותר של מידע לאימון (אחרת הרשת תשנן את המידע)
    \item כיוון שאנחנו מאמנים רק שכבות אחרונות, מספר הפרמטרים המאומנים קטן יותר ולכן יכול להתאים לגודל דאטה קטן
    \item אם מספר הפרמטרים גדול מאד מגודל הדאטה, יש להקפיד על טכניקות להתמודדות עם אפשרות של \ToEn{Overfit} - כגון שימוש ברגולריזציה, \ToEn{Dropout}, ואוגמנטציה.
    \item השימוש באוגמנטציה מאפשר גם אימון ליכולת הפשטה גבוהה יותר וגם כסוג של הגדלה מלאכותית של הדאטה.
\end{itemize}

נקודות וטכניקות שבהן השתמשתי על-מנת לשפר את ביצועי הרשת:

\begin{itemize}
    \item בחירת רשת וטכניקה של כוונון עדין
    \item בחירת מספר אפוקים גדול מצד אחד כדי לתת לרשת אפשרות ללמוד לאורך יותר איטרציות, אבל גם עצירה מוקדמת כדי למנוע ממנה להיתקע.
    \item בחירה נכונה של אלגוריתם \ToEn{Optimizer}. שימוש ב-\ToEn{SGD} סטנדרטי לא הביא לתוצאות טובות. לעומת זאת החלפתו באלגוריתם ממשפחת \ToEn{(Adam, AdamW)} שיפרה בהרבה את התכנסות הרשת והתוצאות הסופיות אליהן הגיעה.
    \item בחירה נכונה של פרמטר \ToEn{Learning Rate}. בניגוד לאינטואיציה הראשונית, קצב לימוד גבוה מדי לא הביא להתכנסות מהירה יותר אלא להיפך, התכנסות איטית לתוצאות לא מוצלחות. דווקא הקטנה שלו הביאה להתכנסות אל תוצאות טובות יותר.
    \item שינוי של גודל האצווה \ToEn{batch}, הקטנתו מאטה את החישובים אבל משפרת את ההתכנסות - בגלל
    \item שינוי דינאמי של קצב הלימוד ע"י שימוש ב- \ToEn{Learning Rate Scheduler}. ה-\ToEn{scheduler} מאפשר להתחיל בקצב לימוד מסויים, ולהקטין אותו שוב ושוב אם הרשת לא מראה שיפור בביצועים. הדבר מאפשר לנצל קצב לימוד גבוה בהתחלה כדי לבצע התכנסות ראשונית, ולאחר מכן שיפור מקומי (שיכול להיות משמעותי מאד) בקצבים נמוכים יותר.
    \item עצירה מוקדמת \ToEn{(Early Stopping)} - הפסקת אימון הרשת אם התוצאות מדגימות שהיא אינה משתפרת יותר, או שהשיפור נעשרה רק על סט האימון, כלומר הרשת מתחילה לבצע \ToEn{Overfitting} ואין טעם להמשיך באימון.
    \item שימוש באוגמנטציות. עם זאת, אוגמנטציות אגרסיביות מדי פגעו מאד בביצועי הרשת - כנראה מפני שבכל זאת כמות הדאטה (בעיקר מחיצת הולידציה) קטנה מדי והרשת מבצעת \ToEn{overfit} על המידע.
\end{itemize}

ב

\lstset{
    basicstyle=\ttfamily\tiny, % Monospaced font, smallest size
    keywordstyle=\bfseries, % Bold keywords
    showstringspaces=false,
    breaklines=true, % Enable line breaking
    breakatwhitespace=true, % Break at whitespace
    breakindent=0pt, % Prevent indentation on new lines
    columns=fullflexible, % Preserve monospaced alignment
    frame=single, % Add a border around the listing
    numbers=none, % Remove line numbers
    xleftmargin=10pt,
    xrightmargin=10pt,
    aboveskip=5pt, % Space above the listing
    belowskip=5pt, % Space below the listing
}

% Define custom replacements for special characters
\newcommand{\vertline}{\textbar} % For vertical line (|)
\newcommand{\cornerline}{\rule{1.5ex}{0.4pt}} % Simulates ─
\newcommand{\branch}{\textbar\cornerline} % Simulates ├─


% Define a verbatim style
\DefineVerbatimEnvironment{CodeBlock}{Verbatim}{
    fontsize=\scriptsize, % Set a smaller font size
    frame=single,         % Add a border
    framesep=2mm,         % Reduce padding around the block
    baselinestretch=1,    % Line spacing
    commandchars=\\\{\},  % Allow \ to escape special characters
}

\newpage
\EN
\begin{verbatim}
Model: RESNET18_2LAYERS
Trainable parameters: ['fc.1.weight', 'fc.1.bias']
============================================================
\end{verbatim}

\begin{minipage}{\linewidth} % Ensure the block doesn't exceed page width
\begin{CodeBlock}
======================================================================================
Layer (type:depth-idx)            Output Shape              Param #     Trainable
======================================================================================
ResNet                            [1, 8]             --                 Partial
├─Conv2d: 1-1                     [1, 64, 128, 128]         (9,408)     False
├─BatchNorm2d: 1-2                [1, 64, 128, 128]         (128)       False
├─ReLU: 1-3                       [1, 64, 128, 128]         --          --
├─MaxPool2d: 1-4                  [1, 64, 64, 64]           --          --
├─Sequential: 1-5                 [1, 64, 64, 64]           --          False
|    └─BasicBlock: 2-1            [1, 64, 64, 64]           (73,984)    False
|    └─BasicBlock: 2-2            [1, 64, 64, 64]           (73,984)    False
├─Sequential: 1-6                 [1, 128, 32, 32]          --          False
|    └─BasicBlock: 2-3            [1, 128, 32, 32]          (230,144)   False
|    └─BasicBlock: 2-4            [1, 128, 32, 32]          (295,424)   False
├─Sequential: 1-7                 [1, 256, 16, 16]          --          False
|    └─BasicBlock: 2-5            [1, 256, 16, 16]          (919,040)   False
|    └─BasicBlock: 2-6            [1, 256, 16, 16]          (1,180,672) False
├─Sequential: 1-8                 [1, 512, 8, 8]            --          False
|    └─BasicBlock: 2-7            [1, 512, 8, 8]            (3,673,088) False
|    └─BasicBlock: 2-8            [1, 512, 8, 8]            (4,720,640) False
├─AdaptiveAvgPool2d: 1-9          [1, 512, 1, 1]            --          --
├─Sequential: 1-10                [1, 8]                    --          True
|    └─Dropout: 2-9               [1, 512]                  --           --
|    └─Linear: 2-10               [1, 8]                    4,104       True
======================================================================================
Total params: 11,180,616
Trainable params: 4,104
Non-trainable params: 11,176,512
Total mult-adds (Units.GIGABYTES): 2.37
=================================================================================
Forward/backward pass size (MB): 51.90
Params size (MB): 44.72
Estimated Total Size (MB): 97.41
===================================================================================
\end{CodeBlock}
\end{minipage}


\newpage
\EN
\centering % Center the block on the page
\begin{verbatim}
Model - RESNET18_5LAYERS
Trainable parameters: 
['layer4.1.conv2.weight', 'layer4.1.bn2.weight', 'layer4.1.bn2.bias', 
'fc.1.weight', 'fc.1.bias']
\end{verbatim}

\begin{minipage}{\linewidth} % Ensure the block doesn't exceed page width
\begin{CodeBlock}
========================================================================================
Layer (type:depth-idx)           Output Shape              Param #           Trainable
========================================================================================
ResNet                           [1, 8]                    --                Partial
├─Conv2d: 1-1                    [1, 64, 128, 128]         (9,408)           False
├─BatchNorm2d: 1-2               [1, 64, 128, 128]         (128)             False
├─ReLU: 1-3                      [1, 64, 128, 128]         --                --
├─MaxPool2d: 1-4                 [1, 64, 64, 64]           --                --
├─Sequential: 1-5                [1, 64, 64, 64]           --                False
|    └─BasicBlock: 2-1           [1, 64, 64, 64]           (73,984)          False
|    └─BasicBlock: 2-2           [1, 64, 64, 64]           (73,984)          False
├─Sequential: 1-6                [1, 128, 32, 32]          --                False
|    └─BasicBlock: 2-3           [1, 128, 32, 32]          (230,144)         False
|    └─BasicBlock: 2-4           [1, 128, 32, 32]          (295,424)         False
├─Sequential: 1-7                [1, 256, 16, 16]          --                False
|    └─BasicBlock: 2-5           [1, 256, 16, 16]          (919,040)         False
|    └─BasicBlock: 2-6           [1, 256, 16, 16]          (1,180,672)       False
├─Sequential: 1-8                [1, 512, 8, 8]            --                Partial
|    └─BasicBlock: 2-7           [1, 512, 8, 8]            (3,673,088)       False
|    └─BasicBlock: 2-8           [1, 512, 8, 8]            4,720,640         Partial
├─AdaptiveAvgPool2d: 1-9         [1, 512, 1, 1]            --                --
├─Sequential: 1-10               [1, 8]                    --                True
|    └─Dropout: 2-9              [1, 512]                  --                --
|    └─Linear: 2-10              [1, 8]                    4,104             True
==========================================================================
Total params: 11,180,616
Trainable params: 2,364,424
Non-trainable params: 8,816,192
Total mult-adds (Units.GIGABYTES): 2.37
======================================================================
Input size (MB): 0.79
Forward/backward pass size (MB): 51.90
Params size (MB): 44.72
Estimated Total Size (MB): 97.41
===========================================================================
\end{CodeBlock}
\end{minipage}
\HE

\EN
\begin{tcolorbox}[colframe=black, colback=gray!5, boxrule=0.5mm, sharp corners]

\begin{minted}{python}

# Define the scheduler
scheduler = ReduceLROnPlateau(
    optimizer,
    mode='min',
    factor=0.5,
    patience=3,
    threshold=1e-5,
    cooldown=2,
    min_lr=1e-6,
)

# Use in the code:

# Step the scheduler with validation loss
scheduler.step(val_precision)

# Print the current learning rate
for param_group in optimizer.param_groups:
    print(f" Learning Rate: {param_group['lr']}")
\end{minted}
\end{tcolorbox}

\HE

\HE

\EN



\end{document}