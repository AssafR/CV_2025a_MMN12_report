% --------------------------------------------------------------
% This is all preamble stuff that you don't have to worry about.
% Head down to where it says "Start here"
% --------------------------------------------------------------
 
\documentclass[11pt,a4paper]{article}

\usepackage{pgfplots} % For plotting
\usepackage{pgfplotstable}
\usepackage{enumitem} % Custom list handling
\usepackage{booktabs} % For better table styling
\usepackage{listings}
\usepackage{xcolor}

\pgfplotsset{compat=1.18} % Ensure compatibility
\usepackage{caption} % For \captionof
\usepackage{mathrsfs,amsmath}   %The amsmath package is included for \xrightarrow
\usepackage[top=2cm,bottom=2cm,left=2.5cm,right=2cm]{geometry}
\usepackage{array} % For better column formatting
\usepackage{ragged2e} % For \RaggedRight to ensure proper alignment
% \usepackage{polyglossia}
\usepackage{framed}


\usepackage[english, bidi=basic, layout=sectioning.tabular]{babel}
\babelprovide[import,main]{hebrew}
\babelfont{sf}
          [Ligatures={Common,Discretionary,TeX}]{Libertinus Sans}
\babelfont{tt}
          [Ligatures=TeX]{Libertinus Mono}


\setmainfont{David CLM}[Script=Hebrew]
%\setmainfont{Noto Sans Hebrew}[Language=Hebrew, Script=Hebrew]

\babelfont[hebrew]{rm}[Script=Hebrew]{David CLM}
\usepackage{fontspec}

\newfontfamily\hebrewfont[Script=Hebrew]{David CLM} %This font will work in Overleaf, Linux and Mac installs, if you're running windows you'd need to pick the right ones
\newfontfamily\hebrewfontsf[Script=Hebrew]{Miriam CLM}
\newfontfamily\englishfont{Times New Roman} % English font
%\newfontfamily\englishfont{TeX Gyre Termes} % Robust alternative to Times New Roman
          
% [Ligatures={Common,Discretionary,TeX}]{Libertinus Serif} % Or any font that supports Hebrew.

\newenvironment{solution}{\begin{proof}[Solution]}{\end{proof}}

\newcommand{\EN}{\selectlanguage{english}}
\newcommand{\HE}{\selectlanguage{hebrew} \vspace{0pt}}
\newcommand{\ToEn}[1]{\EN{#1}\HE}

% Define the custom itemize environment for hebrewlist
\newlist{hebrewlist}{itemize}{1} % Create a new list type
\setlist[hebrewlist,1]{%
    label=\fixNumbering{\thehebrewCounter}, % Use Hebrew counter with tightly bound dot
    ref=\thehebrewCounter, % Reference format (if needed for cross-referencing)
    leftmargin=2em, % Adjust the left margin
    labelsep=1em, % Adjust spacing between the label and the text
    itemsep=0.5\baselineskip, % Adjust spacing between items
    before=\usecounter{hebrewCounter} % Enable the Hebrew counter
}

% Define the macro to switch languages
\newcommand{\switchlang}[2]{%
    \foreignlanguage{#1}{#2} % Switch to the specified language for the given text
}


% Example: Switch to Hebrew for this section
%    \switchlang{hebrew}{זוהי טקסט בעברית}


%--------------------------------


\newcommand{\CreateCustomTable}[2]{%
    \begin{table}[ht!]
    \centering
    \caption{#1}
    \EN
    \pgfplotstabletypeset[
        col sep=comma, % Specifies that the CSV uses commas as the column separator
        header=true, % Use the first row as the table header
        columns/extractor/.style={string type}, % Format 'extractor' column as text
        columns/classifier/.style={string type}, % Display 'classifier' with 4 decimal points
        columns/accuracy/.style={fixed, precision=4}, % Display 'Repeatability' with 4 decimal points
        columns/AUC/.style={fixed, precision=4}, % Display 'LocalizationError' with 4 decimal points
        every head row/.style={before row=\hline, after row=\hline}, % Horizontal lines above and below header
        every last row/.style={after row=\hline} % Add a horizontal line after the last row
    ]{#2}
    \end{table}
}


% Add a figure
\newcommand{\CreateFigure}[2]{%
    \begin{figure}[ht!]
        \centering
        \includegraphics[width=\textwidth]{#1} % Use the first parameter as the filename
        \caption{#2} % Use the second parameter as the caption
        \label{fig:#1} % Optional: Use the filename as the label (or adjust as needed)
    \end{figure}
}


\lstset{
    basicstyle=\ttfamily\footnotesize,  % Use a terminal font and reduce the size
    frame=single,                       % Add a frame around the code
    xleftmargin=1em,                    % Adjust left margin
    showspaces=false,                   % Do not show spaces explicitly
    showstringspaces=false,             % Do not show spaces in strings
    breaklines=true,                    % Enable line breaking
    backgroundcolor=\color{gray!10},    % Light gray background
}



% --------------


% Redefine the itemize bullets globally
\renewcommand\labelitemi{$\bullet$}
\renewcommand\labelitemii{$\circ$} % Open circle for the second level
\renewcommand\labelitemiii{$\star$} % Star for the third level


% Set up a Hebrew-compatible font
\setmainfont{David CLM} % Replace with another Hebrew-compatible font if needed

% Define Hebrew labels manually using a custom counter
\newcounter{hebrewCounter} % Create a new counter
\renewcommand{\thehebrewCounter}{% Map the counter value to Hebrew letters
    \ifcase\value{hebrewCounter}
    \or א
    \or ב
    \or ג
    \or ד
    \or ה
    \or ו
    \or ז
    \or ח
    \or ט
    \or י
    \else ?
    \fi
}

% Force Hebrew letter and period to act as a single unit
\newcommand{\fixNumbering}[1]{\makebox[0pt][r]{#1\kern-0.3em.}} % Use \kern to remove space

\begin{document}

 
% --------------------------------------------------------------
%                         Start here
% --------------------------------------------------------------
\HE 
\title{ממ"ן 12 - ראייה ממוחשבת}
\author{אסף רזון}
\maketitle
בתרגיל זה  3 חלקים. 


\ToEn{train-set} כאשר
\ToEn{60\%} מהתמונות ישמשו לאימון, \ToEn{20\%} לוולידציה ו- \ToEn{20\%} לבדיקה.
\section*{שאלה {1}}
חלקו את הנתונים לנתוני אימון 
\medskip
\subsection*{חלוקה של הדאטה למחיצות}
החלוקה נעשתה תוך שימוש בפונקציית \ToEn{train\_test\_split} מהספרייה, הקוד נמצא בקובץ \ToEn{split\_train\.py}.
לאחר יצירת רשימת כל הקבצים, נעשית חלוקה לשלוש המחיצות \ToEn{Train, Validation, Test} והקבצים מופנים לפי ההסתברויות ל-3 ספריות בשמות המתאימים.

\subsection*{שיטת העבודה}
ניתן לחלק את ה-\ToEn{pipeline} לכמה שלבים, שכל אחד מהם קורא פלט מדיסק, מעבד אותו, ושומר לדיסק את השלב הבא.
\newline
כך ניתן לפתח כל שלב בנפרד ולחזור אחורה בלי להתחיל בכל פעם מההתחלה.

השלבים הם:
\begin{enumerate}[start=0]
\item חלוקה של הדאטה (תמונות) למחיצות
\item הפעלה של \ToEn{Feature Extraction} על כל התמונות ושמירת הפיצ'רים בלבד
\item פעולת קוונטיזציה (\ToEn{Vector Quantization})  - אימון \ToEn{K-Means} על מרחב הפיצ'רים שהתקבל בשלב הקודם ושמירת המודל. מספר המרכזים שבחרתי בשלב זה הוא 128.
\item שלב תרגום ע"י \ToEn{Bag Of Features} - לכל תמונה המאופיינת כעת ע"י סט פיצ'רים, חישוב האשכול \ToEn{(Cluster)} שאליו שייך הפיצ'ר. כעת הייצוג של התמונה הופך להיות היסטוגרמה - לכל אשכול k מהו מספר המאפיינים ששייך לו. כעת הייצוג הוא של כל תמונה כהיסטוגרמה, והייצוג הזה נשמר לדיסק.
\item אימון מסווג אשר מקבל כקלט בשלב האימון סט של היסטוגרמה+מחלקה , ולומד לסווג היסטוגרמה למחלקה
\item הפעלת המסווג על מחיצות \ToEn{test, validation} וחישוב המטריקות המתאימות להערכת הביצועים
\end{enumerate}
\par
נשים לב כי ה-\ToEn{pipeline} הזה זהה לגמרי בין שאלות 1,2 - ההבדל העיקרי הוא איזה אלגוריתם מבצע את פעולת \ToEn{Feature Extraction}.\par
בשאלה מס' 1 השתמשתי באלגוריתם \ToEn{Orb} ובשאלה 2 השתמשתי בשכבות הראשונות של רשת נוירונים מסוג \ToEn{VGG16} ללא השכבות האחרונות שלה - המוצא של שכבות אלה הוא 64 פיצ'רים בגודל 512, שכל אחד מהם מתאים לסביבה של ריבוע אחד בתמונה המקורית ומתאר אותו ואת סביבתו (ברשת קלאסית בגודל \ToEn{64x64} כל סביבה כזאת היא פיקסל).

\subsection*{נקודות חשובות בפתרון של שני הסעיפים האלה}
\begin{itemize}
    \item מספר הפיצ'רים שמחזירה רשת הנוירונים הוא קבוע. אבל מספר הפיצ'רים שמוחזרים ע"י אלגוריתם כגון \ToEn{Orb} יכול להיות קטן יותר או גדול יותר מהמספר שהחלטנו לקבוע (במקרה זה - 500).\newline
    במקרה זה יש לחתוך או להשלים את הפיצ'רים למספר קבוע, כיוון שהיישום הבסיסי של היסטוגרמה כזו מניח מספר קבוע (וסכום ערכי ההיסטוגרמה יהיה שווה למספר הזה). \newline
    לכן חייבים לבצע התאמה למספר קבוע:
    \begin{itemize}
    \item  
אם המספר קטן מדי, הוא מושלם באמצעות אפסים (כלומר פיצ'רים שערך כולם הם אפס).
\item אם המספר גדול מדי, ממיינים את הפיצ'רים לפי ערך ה- \ToEn{Response} ושומרים רק את בעלי הערכים הגבוהים עד למספר הרצוי.
\item ערך זה מייצג "איכות" או "בטחון" של האלגוריתם בנקודת המפתח שהתגלתה. הנקודות שבהן ערך זה גבוה הן הבולטות יותר. לכן הגיוני להשתמש בנקודות שיש להן ערך תגובה גבוה (עד למספר הנדרש) ולהתעלם מן האחרות.
\item אם אכן יהיו הרבה "השלמות" של פיצ'רים ריקים (ערך 0) בסט האימון, סביר להניח שלפחות אחד מהאשכולות שיחושב באלגוריתם \ToEn{k-means} יהיה קרוב מאד אל הנקודה בראשית הצירים, שאליה ימופו כל הקואורדינטות הריקות הנ"ל.
    \end{itemize}
\item המידע הטבלאי נשמר באמצעות \ToEn{Pandas DataFrame} בפורמט \ToEn{feather} המאופיין ביעילות קריאה גבוהה. 
\item המידע הבינארי, כגון סריאליזציה של מודלים, נשמר בפורמט \ToEn{pickle}.
\item לצורך בניית המסווג בשלב הסופי, ניסיתי מספר אפשרויות.
    \begin{itemize}
        \item שתי וריאציות של מסווג \ToEn{AdaBoostClassifier}
        \item שתי וריאציות של מסווג \ToEn{XGBClassifier}
    \end{itemize}
\end{itemize}

\subsection*{הרצת השאלה}
לצורך הרצת פתרון השאלה יש להריץ את הפקודה:
\EN
\begin{framed}
\begin{verbatim}
cd src
python mmn12_q1.py
\end{verbatim}
\end{framed}
\HE
במצב ברירת המחדל ללא פרמטרים, ההרצה תתחיל משלב 1 כפי שמפורט ב"שיטת העבודה".
לחילופין ניתן להריץ החל משלב מתקדם יותר, למשל:
\EN
\begin{framed}
\begin{verbatim}
python mmn12_q1.py --stage=3
\end{verbatim}
\end{framed}
\HE

\section*{שאלות שנשאלנו}
מה הפרמטרים האופטימליים של המסווג?\newline
יש להציג את הביצועים ע"י עקומות 
\ToEn{ROC}, חישוב ה-\ToEn{AUC} וגם \ToEn{confusion matrix} ו- \ToEn{precision-recall} 

\section*{תשובות לשאלות}
הטבלה:

\CreateCustomTable{ביצועי המסווגים השונים}{data/results_extractor_Orb.csv}
\HE
על-פי מדד ה-\ToEn{AUC}, נראה שהמסווג המוצלח ביותר עבור \ToEn{Orb} הוא \ToEn{ADABOOST2}. הפרמטרים שבהם השתמשתי עבורו הם:
\EN
\begin{lstlisting}
'ADABOOST2':
AdaBoostClassifier(
    estimator=DecisionTreeClassifier(max_depth=2), 
    n_estimators=20,  # Number of boosting rounds
    learning_rate=0.2,  # Learning rate
    random_state=42  # Seed for reproducibility
)
\end{lstlisting}
\HE

והגרפים המתאימים עבור מחיצת ה-\ToEn{Test} הם:

\CreateFigure{images/roc_curve_global_Orb_ADABOOST2_test.png}{עקום \ToEn{ROC}}
\CreateFigure{images/prc_curve_global_Orb_ADABOOST2_test.png}{עקום \ToEn{Precision-Recall}}
\CreateFigure{images/confusion_matrix_Orb_ADABOOST2_test.png}{מטריצת הבלבול}




% \CreateCSVTable{data/results_extractor_Orb.csv}{accuracy,AUC}{2}{Results for ORB Extractor}







\newpage
\HE
\section*{שאלה {2}}
בשאלה 2 השתמשתי באותו ה- \ToEn{pipeline} , עם הבדלים מעטים בלבד: שימוש בטנזורים (התוצר שיוצא מתוך הרשת), מימדים שונים במעט, וכו'.

גם כאן בשלב האחרון בדקתי את אותם מסווגים ואלה התוצאות שלהם:

\CreateCustomTable{ביצועי המסווגים השונים}{data/results_extractor_VGG.csv}

כל ארבעת המסווגים הגיעו לתוצאות דומות, אם כי כאן המוצלח ביותר היה \ToEn{XGB2} , אשר הפרמטרים שלו הם:
\EN
\begin{lstlisting}
'XGB2': 
XGBClassifier(
    objective='multi:softmax',  # Multiclass classification
    num_class=8,  # Number of classes
    max_depth=3,  # Tree depth
    learning_rate=0.03,  # Learning rate (eta)
    n_estimators=10,  # Number of trees
    random_state=42  # Seed for reproducibility
    ),
\end{lstlisting}
\HE

ואילו הגרפים שמתאימים לתוצאות שלו הם:



\CreateFigure{images/roc_curve_global_VGG_XGB2_test.png}{עקום \ToEn{ROC}}
\CreateFigure{images/prc_curve_global_VGG_XGB2_test.png}{עקום \ToEn{Precision-Recall}}
\CreateFigure{images/confusion_matrix_VGG_XGB2_test.png}{מטריצת הבלבול}




\newpage
\HE
\section*{שאלה {3}}
בחלק 1 נדרשנו לממש את הגלאים הנ"ל:
\EN


\end{document}