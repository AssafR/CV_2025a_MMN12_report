\usepackage{pgfplots} % For plotting
\usepackage{pgfplotstable}
\usepackage{enumitem} % Custom list handling
\usepackage{booktabs} % For better table styling
\usepackage{listings}
\usepackage{xcolor}

\pgfplotsset{compat=1.18} % Ensure compatibility
\usepackage{caption} % For \captionof
\usepackage{mathrsfs,amsmath}   %The amsmath package is included for \xrightarrow
\usepackage[top=2cm,bottom=2cm,left=2.5cm,right=2cm]{geometry}
\usepackage{array} % For better column formatting
\usepackage{ragged2e} % For \RaggedRight to ensure proper alignment
% \usepackage{polyglossia}
\usepackage{framed}
\usepackage{listings}
\usepackage{fancyvrb} % For better verbatim rendering


\usepackage{fontspec}
\setmonofont{DejaVu Sans Mono}[
    Scale=MatchLowercase,
    BoldFont={DejaVu Sans Mono Bold},
    ItalicFont={DejaVu Sans Mono Oblique}, % Use "Oblique" for italic
    BoldItalicFont={DejaVu Sans Mono Bold Oblique}
]

\usepackage{minted}
\usepackage{tcolorbox} % Add this to the preamble

\usepackage[english, bidi=basic, layout=sectioning.tabular]{babel}
\babelprovide[import,main]{hebrew}
\babelfont{sf}
          [Ligatures={Common,Discretionary,TeX}]{Libertinus Sans}
\babelfont{tt}
          [Ligatures=TeX]{DejaVu Sans Mono} % Replace Fira Mono here as well



\setmainfont{David CLM}[Script=Hebrew]
%\setmainfont{Noto Sans Hebrew}[Language=Hebrew, Script=Hebrew]

\babelfont[hebrew]{rm}[Script=Hebrew]{David CLM}

\newfontfamily\hebrewfont[Script=Hebrew]{David CLM} %This font will work in Overleaf, Linux and Mac installs, if you're running windows you'd need to pick the right ones
\newfontfamily\hebrewfontsf[Script=Hebrew]{Miriam CLM}
\newfontfamily\englishfont{Times New Roman} % English font
%\newfontfamily\englishfont{TeX Gyre Termes} % Robust alternative to Times New Roman
          
% [Ligatures={Common,Discretionary,TeX}]{Libertinus Serif} % Or any font that supports Hebrew.

\newenvironment{solution}{\begin{proof}[Solution]}{\end{proof}}

\newcommand{\EN}{\selectlanguage{english}}
\newcommand{\HE}{\selectlanguage{hebrew} \vspace{0pt}}
\newcommand{\ToEn}[1]{\EN{#1}\HE}

% Define the custom itemize environment for hebrewlist
\newlist{hebrewlist}{itemize}{1} % Create a new list type
\setlist[hebrewlist,1]{%
    label=\fixNumbering{\thehebrewCounter}, % Use Hebrew counter with tightly bound dot
    ref=\thehebrewCounter, % Reference format (if needed for cross-referencing)
    leftmargin=2em, % Adjust the left margin
    labelsep=1em, % Adjust spacing between the label and the text
    itemsep=0.5\baselineskip, % Adjust spacing between items
    before=\usecounter{hebrewCounter} % Enable the Hebrew counter
}

% Define the macro to switch languages
\newcommand{\switchlang}[2]{%
    \foreignlanguage{#1}{#2} % Switch to the specified language for the given text
}


% Example: Switch to Hebrew for this section
%    \switchlang{hebrew}{זוהי טקסט בעברית}


%--------------------------------


\newcommand{\CreateCustomTable}[2]{%
    \begin{table}[ht!]
    \centering
    \caption{#1}
    \EN
    \pgfplotstabletypeset[
        col sep=comma, % Specifies that the CSV uses commas as the column separator
        header=true, % Use the first row as the table header
        columns/extractor/.style={string type}, % Format 'extractor' column as text
        columns/classifier/.style={string type}, % Display 'classifier' with 4 decimal points
        columns/accuracy/.style={fixed, precision=4}, % Display 'Repeatability' with 4 decimal points
        columns/AUC/.style={fixed, precision=4}, % Display 'LocalizationError' with 4 decimal points
        every head row/.style={before row=\hline, after row=\hline}, % Horizontal lines above and below header
        every last row/.style={after row=\hline} % Add a horizontal line after the last row
    ]{#2}
    \end{table}
}


% Add a figure
\newcommand{\CreateFigure}[2]{%
    \begin{figure}[ht!]
        \centering
        % Adjust the spacing before and after the caption
        \setlength{\abovecaptionskip}{2pt} % Reduce space above the caption
        \setlength{\belowcaptionskip}{2pt} % Reduce space below the caption
        \includegraphics[width=0.52\textwidth]{#1} % Increase the width to 55% of text width
        \caption{#2} % Use the second parameter as the caption
        \label{fig:#1} % Optional: Use the filename as the label (or adjust as needed)
    \end{figure}
}



\lstset{
    basicstyle=\ttfamily\footnotesize,  % Use a terminal font and reduce the size
    frame=single,                       % Add a frame around the code
    xleftmargin=1em,                    % Adjust left margin
    showspaces=false,                   % Do not show spaces explicitly
    showstringspaces=false,             % Do not show spaces in strings
    breaklines=true,                    % Enable line breaking
    backgroundcolor=\color{gray!10},    % Light gray background
}



% --------------


% Redefine the itemize bullets globally
\renewcommand\labelitemi{$\bullet$}
\renewcommand\labelitemii{$\circ$} % Open circle for the second level
\renewcommand\labelitemiii{$\star$} % Star for the third level


% Set up a Hebrew-compatible font
\setmainfont{David CLM} % Replace with another Hebrew-compatible font if needed

% Define Hebrew labels manually using a custom counter
\newcounter{hebrewCounter} % Create a new counter
\renewcommand{\thehebrewCounter}{% Map the counter value to Hebrew letters
    \ifcase\value{hebrewCounter}
    \or א
    \or ב
    \or ג
    \or ד
    \or ה
    \or ו
    \or ז
    \or ח
    \or ט
    \or י
    \else ?
    \fi
}

% Force Hebrew letter and period to act as a single unit
\newcommand{\fixNumbering}[1]{\makebox[0pt][r]{#1\kern-0.3em.}} % Use \kern to remove space
