\HE
\section*{שאלה {3}}
בשאלה 3 התבקשנו לממש רשת נוירונים מסוג קונבולוציה ולאמן אותה כדי לסווג את הדאטה ל-8 המחלקות שלו.\newline
להרצה:
\EN
\begin{framed}
\begin{verbatim}
cd src
python mmn12_q3.py
\end{verbatim}
\end{framed}
\HE

כצעד ראשון התבקשנו להגדיר \ToEn{dataloader} ולהציג מספר דוגמאות מכל מחלקה.
תמונת הדוגמאות נוצרת בשלב מס' 1 של השאלה השלישית. 
\CreateFigure{images/class_sample_images.jpg}{דוגמאות מתוך סט האימון עבור כל אחת מ-8 המחלקות שאותן נדרשים לסווג}
\newline

\subsection*{שיטת העבודה}
על-מנת להגיע לתוצאות הנדרשות (90 אחוזי דיוק - הבנתי זאת בתור \ToEn{precision}) נדרשו החלטות וניסויים רבים.

החלטתי לבחור בטכניקה של \ToEn{Fine-Tuning} על רשת קיימת.
במצב כזה, הרשת שאומנה על כמויות גדולות של דאטה (תמונות) למדה לחלץ פיצ'רים רלוונטיים למשימה של זיהוי וסיווג תמונות. נשאר רק לבצע כוונון עדין יותר עבור המטרה הספציפית הזאת.\newline
כדי לבצע זאת "מקפיאים" את השכבות הראשונות ברשת, כלומר מונעים מהן להתעדכן גם בשלב האימון (לא מתבצע עליהן \ToEn{Backpropagation} וחישובי גראדיינטים). לעומת זאת השכבות האחרונות, בד"כ שכבת \ToEn{Fully Connected}, כן לומדות את הדאטה החדש ומתכווננות לפיו. \newline

\subsection*{יתרונות נוספים של גישה זאת:}
\begin{itemize}
    \item הרשת גדולה ולכן יכולה להתמודד עם משימות מתוחכמות
    \item נשים לב שהדאטה שלנו קטן יחסית, ולכן לא מספיק לאימון מלא של רשת גדולה. באופן כללי מספר פרמטרים גדול דורש כמות גדולה יותר של מידע לאימון (אחרת הרשת תשנן את המידע)
    \item כיוון שאנחנו מאמנים רק שכבות אחרונות, מספר הפרמטרים המאומנים קטן יותר ולכן יכול להתאים לגודל דאטה קטן
    \item אם מספר הפרמטרים גדול מאד מגודל הדאטה, יש להקפיד על טכניקות להתמודדות עם אפשרות של \ToEn{Overfit} - כגון שימוש ברגולריזציה, \ToEn{Dropout}, ואוגמנטציה.
    \item השימוש באוגמנטציה מאפשר גם אימון ליכולת הפשטה גבוהה יותר וגם כסוג של הגדלה מלאכותית של הדאטה.
\end{itemize}

נקודות וטכניקות שבהן השתמשתי על-מנת לשפר את ביצועי הרשת:

\begin{itemize}
    \item בחירת רשת וטכניקה של כוונון עדין
    \item בחירת מספר אפוקים גדול מצד אחד כדי לתת לרשת אפשרות ללמוד לאורך יותר איטרציות, אבל גם עצירה מוקדמת כדי למנוע ממנה להיתקע.
    \item בחירה נכונה של אלגוריתם \ToEn{Optimizer}. שימוש ב-\ToEn{SGD} סטנדרטי לא הביא לתוצאות טובות. לעומת זאת החלפתו באלגוריתם ממשפחת \ToEn{(Adam, AdamW)} שיפרה בהרבה את התכנסות הרשת והתוצאות הסופיות אליהן הגיעה.
    \item בחירה נכונה של פרמטר \ToEn{Learning Rate}. בניגוד לאינטואיציה הראשונית, קצב לימוד גבוה מדי לא הביא להתכנסות מהירה יותר אלא להיפך, התכנסות איטית לתוצאות לא מוצלחות. דווקא הקטנה שלו הביאה להתכנסות אל תוצאות טובות יותר.
    \item שינוי של גודל האצווה \ToEn{(batch)}, הקטנתו מאטה את החישובים אבל משפרת את ההתכנסות - בגלל
    \item שינוי דינאמי של קצב הלימוד ע"י שימוש ב- \ToEn{Learning Rate Scheduler}. ה-\ToEn{scheduler} מאפשר להתחיל בקצב לימוד מסויים, ולהקטין אותו שוב ושוב אם הרשת לא מראה שיפור בביצועים. הדבר מאפשר לנצל קצב לימוד גבוה בהתחלה כדי לבצע התכנסות ראשונית, ולאחר מכן שיפור מקומי (שיכול להיות משמעותי מאד) בקצבים נמוכים יותר.
    \item עצירה מוקדמת \ToEn{(Early Stopping)} - הפסקת אימון הרשת אם התוצאות מדגימות שהיא אינה משתפרת יותר, או שהשיפור נעשרה רק על סט האימון, כלומר הרשת מתחילה לבצע \ToEn{Overfitting} ואין טעם להמשיך באימון.
    \item שימוש באוגמנטציות. עם זאת, אוגמנטציות אגרסיביות מדי פגעו מאד בביצועי הרשת - כנראה מפני שבכל זאת כמות הדאטה (בעיקר מחיצת הולידציה) קטנה מדי והרשת מבצעת \ToEn{overfit} על המידע. למשל פעולת \ToEn{GaussianBlur} שאמורה לסייע במקרים כלליים, רק הפריעה על הדאטה הקטן הספציפי שלנו.
\end{itemize}

הפרמטרים של \ToEn{Learning Scheduler} שנבחרו הם כנ"ל:
\EN
\begin{tcolorbox}[colframe=black, colback=gray!5, boxrule=0.5mm, sharp corners]
\begin{minted}{python}
# Define the scheduler
scheduler = ReduceLROnPlateau(
    optimizer,
    mode='min',
    factor=0.5,
    patience=3,
    threshold=1e-5,
    cooldown=2,
    min_lr=1e-6,
)

# Use in the code:

# Step the scheduler with validation loss
scheduler.step(val_precision)

# Print the current learning rate
for param_group in optimizer.param_groups:
    print(f" Learning Rate: {param_group['lr']}")
\end{minted}
\end{tcolorbox}
\HE


\HE 
\subsection*{דוגמה לפלט של הרשת בשלב סיום האימון:}
\EN
\begin{tcolorbox}[colframe=black, colback=gray!5, boxrule=0.5mm, sharp corners]
\begin{verbatim}
 Learning Rate: 0.00025

Epoch 8/50                                                    
 Training  : 100%|█████████| 214/214 [00:02<00:00, 91.56it/s] 
 Validation: 100%|██████████| 76/76 [00:00<00:00, 125.22it/s] 
 Training Loss:      0.2174, Validation Loss:      0.2102     
 Training Precision: 0.9239, Validation Precision: 0.9287     
 Learning Rate: 0.00025                                       
Early stopping triggered. Patience: 2                         
Metrics saved to ../report/data/RESNET18_5LAYERS.csv          
Loss plot saved to ../report/images/RESNET18_5LAYERS.png      
-------------------                                           
Final on best model:                                          
 Validation: 100%|██████████| 76/76 [00:00<00:00, 123.19it/s] 
 Test      : 100%|██████████| 75/75 [00:00<00:00, 119.60it/s] 
 Train     : 100%|████████| 214/214 [00:01<00:00, 122.16it/s] 
Validation Loss:           0.2067, Train Loss     : 0.0454    
Validation precision:      0.9270, Train Precision: 0.9906    
Test       precision:      0.9146                             
End                                                           
\end{verbatim}
\end{tcolorbox}

\HE
בעמוד הבא ניתן לראות עבור שתי רשתות שנבדקו, מהו מבנה הרשת - כולל אילו שכבות נשארו לא מוקפאות ומה מספר הפרמטרים שלהן.\newline
התוצאות הסופיות שנבחרו הן מהרשת \ToEn{RESNET18\_5LAYERS} , כלומר רשת המבוססת על המשקלות של \ToEn{Resnet18}, אבל עם חמש שכבות אחרונות שניתנות לאימון. שכבות אלה שקולות לבלוק הבסיסי בשכבה הסדרתית האחרונה ולאחריה החלק הלינארי.\newline
בגרסת \ToEn{RESNET18\_2LAYERS}, רק השכבה הלינארית האחרונה היא זו שמתאמנת.

\lstset{
    basicstyle=\ttfamily\tiny, % Monospaced font, smallest size
    keywordstyle=\bfseries, % Bold keywords
    showstringspaces=false,
    breaklines=true, % Enable line breaking
    breakatwhitespace=true, % Break at whitespace
    breakindent=0pt, % Prevent indentation on new lines
    columns=fullflexible, % Preserve monospaced alignment
    frame=single, % Add a border around the listing
    numbers=none, % Remove line numbers
    xleftmargin=10pt,
    xrightmargin=10pt,
    aboveskip=5pt, % Space above the listing
    belowskip=5pt, % Space below the listing
}

% Define custom replacements for special characters
\newcommand{\vertline}{\textbar} % For vertical line (|)
\newcommand{\cornerline}{\rule{1.5ex}{0.4pt}} % Simulates ─
\newcommand{\branch}{\textbar\cornerline} % Simulates ├─


% Define a verbatim style
\DefineVerbatimEnvironment{CodeBlock}{Verbatim}{
    fontsize=\scriptsize, % Set a smaller font size
    frame=single,         % Add a border
    framesep=2mm,         % Reduce padding around the block
    baselinestretch=1,    % Line spacing
    commandchars=\\\{\},  % Allow \ to escape special characters
}

\newpage
\EN
\begin{verbatim}
Model: RESNET18_2LAYERS
Trainable parameters: ['fc.1.weight', 'fc.1.bias']
============================================================
\end{verbatim}

\begin{minipage}{\linewidth} % Ensure the block doesn't exceed page width
\begin{CodeBlock}
======================================================================================
Layer (type:depth-idx)            Output Shape              Param #     Trainable
======================================================================================
ResNet                            [1, 8]             --                 Partial
├─Conv2d: 1-1                     [1, 64, 128, 128]         (9,408)     False
├─BatchNorm2d: 1-2                [1, 64, 128, 128]         (128)       False
├─ReLU: 1-3                       [1, 64, 128, 128]         --          --
├─MaxPool2d: 1-4                  [1, 64, 64, 64]           --          --
├─Sequential: 1-5                 [1, 64, 64, 64]           --          False
|    └─BasicBlock: 2-1            [1, 64, 64, 64]           (73,984)    False
|    └─BasicBlock: 2-2            [1, 64, 64, 64]           (73,984)    False
├─Sequential: 1-6                 [1, 128, 32, 32]          --          False
|    └─BasicBlock: 2-3            [1, 128, 32, 32]          (230,144)   False
|    └─BasicBlock: 2-4            [1, 128, 32, 32]          (295,424)   False
├─Sequential: 1-7                 [1, 256, 16, 16]          --          False
|    └─BasicBlock: 2-5            [1, 256, 16, 16]          (919,040)   False
|    └─BasicBlock: 2-6            [1, 256, 16, 16]          (1,180,672) False
├─Sequential: 1-8                 [1, 512, 8, 8]            --          False
|    └─BasicBlock: 2-7            [1, 512, 8, 8]            (3,673,088) False
|    └─BasicBlock: 2-8            [1, 512, 8, 8]            (4,720,640) False
├─AdaptiveAvgPool2d: 1-9          [1, 512, 1, 1]            --          --
├─Sequential: 1-10                [1, 8]                    --          True
|    └─Dropout: 2-9               [1, 512]                  --           --
|    └─Linear: 2-10               [1, 8]                    4,104       True
======================================================================================
Total params: 11,180,616
Trainable params: 4,104
Non-trainable params: 11,176,512
Total mult-adds (Units.GIGABYTES): 2.37
=================================================================================
Forward/backward pass size (MB): 51.90
Params size (MB): 44.72
Estimated Total Size (MB): 97.41
===================================================================================
\end{CodeBlock}
\end{minipage}


\newpage
\EN
\centering % Center the block on the page
\begin{verbatim}
Model - RESNET18_5LAYERS
Trainable parameters: 
['layer4.1.conv2.weight', 'layer4.1.bn2.weight', 'layer4.1.bn2.bias', 
'fc.1.weight', 'fc.1.bias']
\end{verbatim}

\begin{minipage}{\linewidth} % Ensure the block doesn't exceed page width
\begin{CodeBlock}
========================================================================================
Layer (type:depth-idx)           Output Shape              Param #           Trainable
========================================================================================
ResNet                           [1, 8]                    --                Partial
├─Conv2d: 1-1                    [1, 64, 128, 128]         (9,408)           False
├─BatchNorm2d: 1-2               [1, 64, 128, 128]         (128)             False
├─ReLU: 1-3                      [1, 64, 128, 128]         --                --
├─MaxPool2d: 1-4                 [1, 64, 64, 64]           --                --
├─Sequential: 1-5                [1, 64, 64, 64]           --                False
|    └─BasicBlock: 2-1           [1, 64, 64, 64]           (73,984)          False
|    └─BasicBlock: 2-2           [1, 64, 64, 64]           (73,984)          False
├─Sequential: 1-6                [1, 128, 32, 32]          --                False
|    └─BasicBlock: 2-3           [1, 128, 32, 32]          (230,144)         False
|    └─BasicBlock: 2-4           [1, 128, 32, 32]          (295,424)         False
├─Sequential: 1-7                [1, 256, 16, 16]          --                False
|    └─BasicBlock: 2-5           [1, 256, 16, 16]          (919,040)         False
|    └─BasicBlock: 2-6           [1, 256, 16, 16]          (1,180,672)       False
├─Sequential: 1-8                [1, 512, 8, 8]            --                Partial
|    └─BasicBlock: 2-7           [1, 512, 8, 8]            (3,673,088)       False
|    └─BasicBlock: 2-8           [1, 512, 8, 8]            4,720,640         Partial
├─AdaptiveAvgPool2d: 1-9         [1, 512, 1, 1]            --                --
├─Sequential: 1-10               [1, 8]                    --                True
|    └─Dropout: 2-9              [1, 512]                  --                --
|    └─Linear: 2-10              [1, 8]                    4,104             True
==========================================================================
Total params: 11,180,616
Trainable params: 2,364,424
Non-trainable params: 8,816,192
Total mult-adds (Units.GIGABYTES): 2.37
======================================================================
Input size (MB): 0.79
Forward/backward pass size (MB): 51.90
Params size (MB): 44.72
Estimated Total Size (MB): 97.41
===========================================================================
\end{CodeBlock}
\end{minipage}
\HE

\EN

\HE
\subsection*{הציגו גרף של ערכי ה- \ToEn{loss}על סט הוולידציה והאימון עבור מקרה של \ToEn{over-fitting}.}
במקרה הזה יש \ToEn{over-fitting} משילוב של הסיבות הבאות:
\begin{itemize}
    \item רשת פשוטה מדי \ToEn{(VGG16)}
    \item אוגמנטציות עדינות מאד
    \item תהליך אימון ממושך מדי ללא \ToEn{Early Stopping}
\end{itemize}

\CreateFigure{images/VGG16.png}{דוגמה להתקדמות \ToEn{Train/Validation} לאורך תהליך האימון}

\newpage
\HE
\subsection*{
חשבו את הביצועים על סט הבדיקה. יש להציג את הדיוק, גרף של \ToEn{ROC} וגרף של \ToEn{Precision-Recall}.}
\CreateFigure{images/roc_curve_global_RESNET18_5LAYERS_test.png}{עקום \ToEn{ROC}}
\CreateFigure{images/prc_curve_global_RESNET18_5LAYERS_test.png}{עקום \ToEn{Precision-Recall}}
\CreateFigure{images/confusion_matrix_VGG_XGB2_test.png}{מטריצת הבלבול}
\clearpage

\CreateFigure{images/RESNET18_5LAYERS.png}{דוגמה להתקדמות \ToEn{Train/Validation} לאורך תהליך האימון:}

\HE